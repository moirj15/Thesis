\chapter{Introduction}

Research in the field of soft-body deformation simulation has been ongoing since the 1980's. 
Originating with the seminal paper by Terzopoulos \etal, Elastically Deformable Models 
\cite{Elastically-Deformable-Models}. Since then, the field has 
grown to encompass several areas of research for achieving realistic soft-body deformations. These
areas include Mass-Spring systems and Finite Element Method systems. 
% TODO: reference mass-spring and fem papers
These systems are typically
used in animation, surgery simulation, and video games.

Due to the limitations at the time, systems which simulate soft-body deformations typically didn't
operate at real-time speeds. These limitations aren't as apparent in systems that don't operate at
real-time speeds, such as animation systems. However, systems such as surgery simulators and video
games must operate at real-time speeds. So it's vital that any soft-body deformation simulation that
they make use of are as fast as possible. With research into new techniques and the advent of modern
hardware, real-time systems for soft-body have come into existence. Some examples of real-time 
systems are Stable Real-Time Deformations \cite{Stable-Real-Time-Deformations} and Deformable Object 
Simulation in Virtual Environment \cite{Deformable-Object-Simulation-in-Virtual-Environment}.

Another system which takes a different approach was described in the work done by Müller \etal The 
paper described a system which can perform motion, deformation, and fracture 
simulations in real-time. To achieve this, Müller \etal utilized a voxelized representation of 
meshes, or voxel-meshes, generated at run time \cite{Muller_Teschner_Gross}. These voxel-meshes 
contained references
to the mesh used to create them, so that when a voxel  was modified by some force the portion of 
the mesh that it referenced was also modified. The forces for deformation and motion were calculated 
using rigid-body physics, while the forces for fracturing were calculated using the finite element 
method. 

The remainder of this proposal focuses on an extension to the technique used by Müller \etal
to improve the performance and realism of soft-body deformations. This extension will make use of 
Bézier curves and deformable voxels. The Bézier curves will be used to control the positions of the
mesh's vertices and the control points of the Bézier curves will be placed on the hull of the voxels.
The deformable voxels will allow these control points to move, resulting in the deformation of the 
mesh. In theory, this will result deformations which are less expensive to compute and are more 
accurate.
%TODO: add a paragraph explaining my extension to muller



\section{Background}
% This section should be sufficient for the reader to understand the
% project area and the relevance of your efforts in the world of
% computer science. The description here should be provide the
% motivation to the reader that you are exploring a problem area that is
% relevant to the CS community.

The field of soft-body deformations has been around since the 1980's, originating with Terzopoulos
\etal's seminal paper Elastically Deformable Models \cite{Elastically-Deformable-Models}. Since 
then, an extraordinary amount of work has been done in the field of soft-body deformations. A 
non-comprehensive overview of this work can be seen in the surveys done by Frisken \etal 
\cite{A-Survey-of-Deformable-Modeling-in-Computer-Graphics} and Nealen \etal
\cite{Phyiscally-Based-Deformable-Models-in-Computer-Graphics}.
We'll observe some of the work done in this field, mainly for deformations of solid 
three-dimensional objects, for two types of systems. Systems based on the Finite Element Method 
(FEM) and Mass-Spring systems. 

\subsection{FEM Systems}

The Finite Element Method works by taking an object and subdividing it into a collection of shapes.
Typically, tetrahedral subdivisions are used for 3D objects, although others can be used as well.
Each object which constitutes the subdivision is made up of a series of points, refereed to as nodes.
These nodes are used to construct matrix, one for each object. These matrices are then combined 
based on neighboring nodes to create what is refereed to as the stiffness matrix. The stiffness 
matrix is used to determine how the object reacts to applied forces \cite{FEM-Explenation}.

The earliest application of FEM to soft-body deformations was in the work done by Terzopoulos \etal
\cite{Elastically-Deformable-Models}. Their work was based on previous work in modeling and 
elastic theory. FEM was used to discretize the equations of motion, with 
focus on surfaces, to achieve their deformations. This yielded realistic results for a number of 
simple objects. 
%TODO: add overview of FEM

FEM systems can be used to achieve fairly realistic deformations. However, the calculations that 
need to be done are fairly expensive, so achieving real-time deformations is difficult. Although,
work has been done to improve the performance of this technique.
An early instance of this method achieving near real-time performance was with the work done by 
Witkin et al. \cite{Fast-Animation-and-Control-of-Nonrigid-Structures}. In their work, deformable 
models were constructed from non-rigid pieces using point-to-point attachment constraints. 
%(NEED TO RE-READ THIS)

A method for fast physically accurate simulation for real time animation and interaction was 
introduced in the work done by James \etal \cite{ArtDefo-Accurate-Real-Time}. 
Unlike previously discussed papers, they used a combination of boundary 
integrals and the boundary element method. To achieve real-time performance, values which are 
typically too expensive to calculate at run time are pre computed and stored in a database for 
lookup during run-time. This lead to the performance of a real-time simulation, at the cost of 
memory and the need to pre-compute values.

Another method for real-time simulation using FEM was introduced in the work done by Debunne \etal 
\cite{Dynamic-Real-Time-Deformations-Using} 
using space and time adaptive sampling. In their work, they describe a system which makes use of a 
non-nested multi-resolution tetrahedral mesh. As objects are deformed in their simulation the 
tetrahedral mesh is refined to focus computation on areas with more deformations than others. This 
level of detail system ensured that meshes don’t become too coarse during deformation. They also 
applied this technique to mass-spring systems. However, it was found that this leads to less 
accurate results.


\subsection{Mass-Spring Systems}

%TODO: add overview of mass-spring systems

Mass-Spring systems treat objects as a network of mass-nodes and springs. As the name implies, 
mass-nodes are points with a specified mass. These nodes are connected to other nodes in the object
with a number of springs. These springs are typically treated as ideal weightless springs. The 
configuration of these springs can vary depending on the desired effect. 
% An illustration of this 
% hierarchy can be seen in Figure \ref{}.
Once the mass-nodes and 
spring network is configured, physical interactions can easily and quickly be simulated by applying
hooke's law, \(F_s = kx\), to each node. 

Mass-Spring systems usually make use of several types of springs to create their deformations. For
instance, in the work done by Chen \etal \cite{Physically-Based-Animation-of-Volumetric-Objects} a
combination of structural, shear, and flexion springs are used. Structural springs are used to
simulate deformations that occur under compressive forces, shear springs are used to simulate shear
forces, and flexion springs are used to simulate flexion deformations. %TODO: re-word
These springs are connected to mass-nodes and the stretching and compression of theses springs 
creates the deformations. For the case of this appear, the mass nodes are connected in a voxel 
structure. As forces are applied to the structure, these connections persist, allowing deformations
to correctly propagate across voxels. However, as a drawback the object has to be rendered as a 
series of voxels. The work done by chen \etal was able to run at interactive speeds.

The introduction of general-purpose computing on graphics processing units (GPGPU) presented a way 
to easily improve the performance of Mass-Spring systems. The work provided by Chang \etal 
\cite{Deformable-Object-Simulation-in-Virtual-Environment} 
 introduced a method for computing deformations 
with a mass-spring model on the GPU. The calculations and subsequent rendering of the deformations 
are done in the vertex and fragment shaders. In order to store the velocity, position, and 
connectivity data on the GPU a series of textures are used, two for each data type. This texture 
duplication allows them to overcome the difficulty of not being able to read and write to the same 
texture. This technique also allows them to keep all of the deformation calculations on the GPU 
without the need to send information back to main memory, which is a considerable bottle-neck. Their 
implementation resulted in a simulation which ran in real-time for objects with over 170,000 
springs.

An important part in creating realistic deformations is volume preservation. Volume preservation is 
normally something which is fairly expensive to compute. Vassilev \etal 
\cite{Simulation-of-a-Deformable-Human-Body} 
 attempts to achieve this by introducing a new type of spring, the support spring, to simulate the 
matter within an object. These springs are connected perpendicular to surface vertices with equal 
length. They are also connected to an imaginary frame within the body. These springs then combat any 
compressive forces that might result in inaccurate volume changes. Their approach achieved real-time 
speeds for meshes of approximately 7000 triangles. 

\subsection{muller}
%TODO: add sub-section for muller's paper and it's shortcomings

%TODO: rename to solution and give a more in depth/high level description of the changes i would make
%TODO: add pseudo code here
%TODO: for the hypothesis specificly mention the metrics i'll be measuring
\section{Solution}
% TODO: come back to this, better describe the extension and my thoughts behind it.
To improve upon the work done by Müller \etal, I propose the following additions. Bézier curves 
should be constructed from mesh edges contained within each voxel and the voxels should be 
deformable.

The Bézier curves will be the main source of deformations. They will be constructed from the edges
of the underlying mesh during voxelization. To construct a Bézier curve, a ray will be cast along 
the corresponding edge of the underlying mesh. The intersections between this ray and the voxel will
be the first and last control points for the Bézier curve. The remaining control points will use the
edge's starting and ending points if the whole edge is contained by the voxel. Otherwise, the vertex
contained by the voxel will be the only control point. The resulting Bézier curve will either be a
quadratic or a cubic curve. In order for this Bézier curve to correctly reconstruct the edge that it
was constructed from, the point on the curve that corresponds to the edge's start and end points 
needs to be found. For quadratic Bezier curves, this can be done by taking the original formula and
rewriting it from the form:

\[P = (1 - t)^2P_0 + 2(1 - t)tP_1 + t^2 P_2\]

\begin{center}
To:
\end{center}
  
\[0 = (P_0 - 2P_1 + P_2)t^2 + (-2P_0 + 2P_1)t + (P_0 - P)\]

The quadratic equation can then be used to solve for \(t\). For cubic Bézier curves, the same 
process can be done. First, the equation for the cubic Bézier curve is rewritten from:

\[P = (1 - t)^3P_0 + 3(1 - t)^2tP_1 + 3(1 - t)t^2P_2 + T^3P_3\]

\begin{center}
To:
\end{center}

\[0 = (-P_0 + 3P_1 - 3P_2 + P_3)t^3 + (3P_0 - 6P_1 + 3P_2)t^2 + (-3P_0 + 3P_1)t + (P_0 - P)\]

Then, the cubic equation can be used to solve for the two \(t\) values needed, one for each 
endpoint. The pseudo code that describes the construction of the voxels and Bézier curves can be 
seen in Figure \ref{fig:VoxelizationAlgorithm}.

In order for the shape of the Bézier curves to change their control points must be moved somehow.
The addition of deformable voxels allows this to occur. As voxels are deformed, the control points
which are attached to the hull will be adjusted based on the voxel deformation. A simplified 
two-dimensional example can be seen in Figure \ref{}. Here we see the direction which force is 
applied to the voxel and the resulting voxel deformation. We also see how the deformation modifies
the position of the control point, the resulting change in the Bézier curve, and the modified 
position of the vertices.


% Bezier curves
% shall be constructed using the edges contained within each voxel and their starting and ending
% control points will be placed on the internal faces of the voxel. Second, the voxels will be 
% allowed to deform based on forces applied to each voxel face. These deformations will cause the 
% control points of the Bezier curves to move, resulting in their shape being modified. This shape
% modification will then be used to modify the mesh, creating the deformation.

My hypothesis is that the proposed extension will provide soft-body deformations that are more
realistic and more efficient to compute than the original technique. In theory, the use of the
Bezier curves and deformable voxels should allow for a low resolution voxel-mesh to achieve similar
deformations to that of a high resolution voxel-mesh.

\begin{figure}
  \hrule
  \vspace{6pt}
  \begin{algorithmic}
  \REQUIRE $size$, The size of the voxels
  \REQUIRE $hollow$, Specifies if the resulting voxel-mesh should be hollow
  \STATE Create a bounding box using the mesh's minimum and maximum vertex
  \STATE Populate the bounding box with voxels of uniform size specified by $size$
  \STATE Place each triangle of the mesh into a tight bounding box
  \FOR{For each Voxel $v$}
      \IF{If $v$ intersects a triangle's bounding } 
          \STATE save $v$
      \ENDIF
  \ENDFOR
  \IF{$hollow$ is true}
      \FOR{For each 2D slice of the bounding box}
          \STATE Fill the voxels that are contained within the existing voxels
      \ENDFOR
  \ENDIF
  \FOR{For each Voxel $v$}
      \FOR{For each edge $e$ in $v$}
          \STATE Cast a ray along $e$ in the directions of its starting and end points
          \STATE Use the intersection points as the first and last control points of the Bézier curve for this edge
          \STATE Use the endpoints of $e$ that are contained by $v$ as the remaining control point(s)
          \STATE Calculate the position on the Bézier curve that corresponds to the endpoints of $e$ that are contained by $v$
      \ENDFOR
  \ENDFOR
  \RETURN the voxel-mesh
  \end{algorithmic}
  \hrule
  \vspace{6pt}
  \caption{Voxelization Algorithm Pseudo Code}
 \label{fig:VoxelizationAlgorithm}
\end{figure}

% Summarize what you think the problem is, and what your hypothesis
% is. Here is a small example based on a successful project by Priyanka
% Sinha: ''Using one technique for schema matching does not seem
% adequate. The hypothesis underlying this sproject is that a holistic
% approach to schema matching based on the three techniques described
% earlier would do an effective approach to schema matching.''

% Additional description to circumscribe the work so that the reader
% knows what you plan to do to establish your hypothesis.

% \section{Solution Design and Implementation}
% Describe how you plan to design and implement a solution. 

% You must also describe how you would use your solution to establish the validity of your hypothesis. 
% Explain the measurements you plan to conduct and how these would establish the validity 
% (or invalidity) of your hypothesis.

%TODO: rename to implementation


%% Obviously you need to delete these lines when you have written up your text


% \begin{itemize}
% \item{} Background: should be sufficient for the reader understand the rest of the report, but 
% perhaps not too long to put the reader to sleep.
% \item{} Basic problem definition and motivation
% \item{} Approaches used to solve the problem (related work)
% \item{} Hypothesis: what you think the problem is and how your solution 
% approach will address the problem
% \item{} Roadmap: how the rest of your report is laid out 
% \end{itemize}

% And yes, this is how you cite a book by Silberschatz~\cite{Silberschatz05-text} or a paper by Dumont~\cite{Dumont2007-robots}.

% And here are examples of how to include figures and tables in the text. Please note that the captions go below for figures and above for tables.


% \begin{figure}[ht]
% \begin{center}
% \includegraphics[width=3.0in]{cs-logo.jpg}
% \end{center}
% \caption{The CS Logo is Above}
% \label{fig:lab}
% \end{figure}


% \begin{table}[h]
% \caption{The Dog Table is Below}
%   \begin{tabular}{ | l | c | r | }
%     \hline
%     tag & breed & age \\
%     \hline \hline
%     13 & Fido & 2 \\
%     \hline
%     14 & Fifi & 4 \\
%     \hline
%   \end{tabular}
% \end{table}

% For both tables and figures, the optional argument controls
% placement as shown:
% \begin{itemize}
%   \item{} h is Here, i.e., the position in the text where the table environment appears.
%   \item{} t is Top, i.e., the top of a text page.
%   \item{} b is Bottom, i.e., at the bottom of a text page.
%   \item{} p is Page of floats, i.e., on a separate float page,
%     which is a page containing no text, only floats.
% \end{itemize}

% Anyway, you can find some easy tutorials on \LaTeX{}.